\documentclass[12pt]{article}
\usepackage{epsfig}
\usepackage{natbib}
\usepackage{multirow}
\bibliographystyle{mybst}
\input util.tex
\bigpage
\newtheorem{theorem}{Theorem}
\begin{document}

\baselineskip=24truept



\title{Note on Kate's Rotation Project }

\date{February 13, 2019} \maketitle


For the linear mixed-effects model to fit the parabolas, we can consider
$$\by = \bX(\bbeta +\bb +\bGamma(\balpha'\bx)) + \bepsilon.$$
The goal is to find a vector $\balpha$ that gives the best ``purity'' when clustering.  $\bX$ will be the design matrix for the orthogonal polynomials; $\bb$ the vector of 
random effects; $\bGamma$ a vector (of the same dimension of $\bbeta$) of fixed-effects.

Actually, we can fit a model separately for the drug and placebo treatments
$$\by_1 = \bX(\bbeta_1 +\bb_1 +\bGamma_1(\balpha'\bx)) + \bepsilon_1\;\;\mbox{and}
\;\;
\by_2 = \bX(\bbeta_2 +\bb_2 +\bGamma_2(\balpha'\bx)) + \bepsilon_2.
$$
My intuition says to find $\balpha$ that leads to big differences in $\bGamma_1$ and $\bGamma_2$ as well as differences between $\bbeta_1$
and $\bbeta_2$.  If we choose $\balpha$ and then fit the model, that will determine $\hat{\bGamma}_1, \hat{\bGamma}_2, \hat{\bbeta}_1$ 
and $\hat{\bbeta}_2$.   


\bibliography{all}

\end{document}

